\documentclass[a4paper,english, 10pt, twoside]{article}
\usepackage[utf8]{inputenc}
\usepackage[T1]{fontenc}
\usepackage[english]{babel}
\usepackage{epsfig}
\usepackage{graphicx}
\usepackage{amsfonts, amssymb, amsmath}
\usepackage{listings}
\usepackage{float}
\usepackage[top=2cm, bottom=2cm, left=2cm, right=2cm]{geometry}
%opening
\title{1D wave equation with finite elements\\ INF5620}
\author{Fredrik E Pettersen}

\begin{document}

\maketitle
The purpose of this project is to derive and analyze a finite element method for the 1D wave equation
\begin{align*}
 \frac{\partial^2 u}{\partial t^2} = c^2\frac{\partial^2 u}{\partial x^2} \Leftrightarrow D_tD_tu = c^2D_xD_xu\\
 x\in[0,L], t\in(0,T]
\end{align*}
with initial and boundary conditions
$$
u(0,t) = U_0(t), u_x(L) = 0, u(x,0) = I(0), u_t(x,0) = V(x)
$$

\section*{a)}
If we now use a finite difference method on this equation we can formulate a series of spatial problems as follows
\begin{align*}
 D_tD_tu = u_{tt} = \frac{u_j^{n+1} - 2u_j^n+u_j^{n-1}}{\Delta t^2} \\
 \frac{u_j^{n+1} - 2u_j^n+u_j^{n-1}}{\Delta t^2} = c^2u_{xx}^n \\
 u_j^{n+1} = \Delta t^2c^2u_{xx}^n + 2u_j^n - u_j^{n-1}
\end{align*}
where $n = 0,1,...,N_x$
\section*{b)}
Using the Galerkin method we can now transform the series of spatial problems to variational form. I have used the testfunction $v(x)$ which is 
part of the function space of P1 elements. Assume (for later) that
\begin{equation}\label{approx}
v(x) = \phi_i, (i=0,1,...,N) \text{  and  } u^n \simeq \sum\limits_{j=0}^{N_x} \phi_jc^n_j
\end{equation}

We want the inner product $\left(R,v\right) = 0 $ where $R =  u_j^{n+1} - \Delta t^2c^2u_{xx}^n - 2u_j^n + u_j^{n-1} = 0$. This means that
\begin{align*}
 \left(R,v\right)=  \left(u_j^{n+1} - \Delta t^2c^2u_{xx}^n - 2u_j^n + u_j^{n-1},v\right) \\
 = \int\limits_0^L \left(u_j^{n+1} - \Delta t^2c^2u_{xx}^n - 2u_j^n + u_j^{n-1}\right)\cdot v dx \\
 = \int\limits_0^Lu_j^{n+1}vdx -2\int\limits_0^Lu_j^nvdx + \int\limits_0^Lu_j^{n-1}vdx - \underbrace{\Delta t^2c^2\int\limits_0^Lu_{xx}^nvdx}_{
 = \Delta t^2c^2\left([u_xv]_0^L-\int\limits_0^Lu_x^nv_xdx\right)}\\
 = \int\limits_0^Lu_j^{n+1}vdx -2\int\limits_0^Lu_j^nvdx + \int\limits_0^Lu_j^{n-1}vdx + \int\limits_0^Lu_x^nv_xdx = 0
\end{align*}
which leads us to the final variational form
$$
\left(u^{n+1},v\right) -2 \left(u^n,v\right) + \left(u^{n-1},v\right) - C^2\left(u^n_x,v_x\right) = 0,\;\;\; \forall \;v\in V
$$

\section*{c)}
Using the approximation of u in (\ref{approx}) we can set up the variational for as a linear system
\begin{align*}
 \int\limits_0^Lu_j^{n+1}vdx -2\int\limits_0^Lu_j^nvdx + \int\limits_0^Lu_j^{n-1}vdx &+ \int\limits_0^Lu_x^nv_xdx    = \\
\sum\limits_j^{N_x}\int\limits_0^L(\phi_j\phi_i)c_j^{n+1}dx - 2\sum\limits_j^{N_x}\int\limits_0^L(\phi_j\phi_i)c_j^{n}dx &+ 
  \sum\limits_j^{N_x}\int\limits_0^L(\phi_j\phi_i)c_j^{n-1}dx - C^2\sum\limits_j^{N_x}\int\limits_0^L(\phi'_j\phi'_i)c_j^{n}dx
\end{align*}
If we now set $M_{ij} = \int\limits_0^L(\phi_j\phi_i)dx$ and $K_{ij} = \int\limits_0^L(\phi'_j\phi'_i)dx $
\end{document}
