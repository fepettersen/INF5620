\documentclass[a4paper,english, 10pt, twoside]{article}
\usepackage[utf8]{inputenc}
\usepackage[T1]{fontenc}
\usepackage[english]{babel}
\usepackage{epsfig}
\usepackage{graphicx}
\usepackage{amsfonts, amssymb, amsmath}
\usepackage{listings}
\usepackage{float}	%force figures in place with command \begin{figure}[H]
\renewcommand{\d}{\partial}
\usepackage[top=2cm, bottom=2cm, left=2cm, right=2cm]{geometry}


%opening
\title{Nonlinear diffusion equation with finite elements}
\author{Fredrik E Pettersen\\ fredriep@student.matnat.uio.no}

\begin{document}

\maketitle

% \begin{abstract}
% In this project we will look at a simple linear second order differential equation.
% \end{abstract}

\section*{About the problem}
In this project we will solve a nonlinear diffusion equation using the finite element method and the FEniCS software package.
More precicely we will be looking at the equation
\begin{equation}\label{eq}
 \rho u_t = \nabla\cdot(\alpha(u)\nabla u) +f(\mathbf{x},t)
\end{equation}
where $\rho$ is a constant, $\alpha(u)$ is a known function of u, and where we have the initial condition $u(\mathbf{x},0) = I(\mathbf{x})$, 
and Neuman boundary conditions $\frac{\d u}{\d n} = 0$.

\section*{Discretization}
In order to solve equation (\ref{eq}) we will need to discretize it. The standard approach is then to pick a nummerical approximation to the 
derivatives. We will use a finite difference discertization in time, more specifically the Backward Euler (BE) discretization, and a finite element 
approximation in space. The BE discertization gives us
\begin{align*}
 \rho u^n = \Delta t\nabla\cdot(\alpha(u)\nabla u) +\Delta t f(\mathbf{x},t) +\rho u^{n-1} \\
 \rho u^n - \Delta t\nabla\cdot(\alpha(u)\nabla u) -\Delta t f(\mathbf{x},t) -\rho u^{n-1} = R
\end{align*}
we now approximate R on a functionspace $V = span\{\phi_i\}$, where $\phi_i$ denotes the P1 elements, and minimize the error by demanding
$$
\big(R,v\big) = 0 \;\;\; \forall \,v\in V
$$
which gives us 
\begin{align*}
  \rho (u^n,v)  -\Delta t (f(\mathbf{x},t_n),v) -\rho (u^{n-1},v) - \Delta t(\nabla\cdot(\alpha(u)\nabla u^n),v)= 0 
\end{align*}
The first three terms are ok for now, but the last term has a double derivative in $u$, which is something we dont want seeing as we are trying to 
approximate $u$ by P1 elements as $u \simeq \sum\limits_{k}u_k\phi_k$, and the double derivative of a linear function is simply zero. To get around 
this problem we try to integrate by parts
\begin{align*}
 \Delta t(\nabla\cdot(\alpha(u)\nabla u^n),v) = \Delta t\left(\int_\Omega \nabla\cdot(\alpha(u)\nabla u^n)vdx\right)\\
 =\Delta t \left(\left[\alpha(u)\nabla u v\right]_{\d \Omega}-\int_\Omega \alpha(u)\nabla u^n\nabla vdx\right) = -\Delta t\int_\Omega 
 \alpha(u)\nabla u^n\nabla vdx
\end{align*}
where we have inserted for the boundary conditions $\frac{\d u}{\d n} = \mathbf{n}\cdot\nabla u = 0$ on $\d\Omega$. We have now arrived on a 
linear variational problem to be solved at each timestep
\begin{align*}
  \rho (u^n,v)  -\Delta t (f(\mathbf{x},t_n),v) -\rho (u^{n-1},v) + \Delta t(\alpha(u)\nabla u^n,\nabla v)= 0  \\
   \underbrace{\rho (u^n,v)   + \Delta t(\alpha(u)\nabla u^n,\nabla v)}_{a(u,v)}=  \underbrace{\Delta t (f(\mathbf{x},t_n),v) +\rho 
   (u^{n-1},v)}_{L(u,v)}
\end{align*}
Notice that we have a nonlinearity in $\alpha(u)$. There are several way to get around this nonlinearity, we will use Picard iteration meaning 
that we start out with $\alpha(u^{n-1})$ and solve the linear system to get a solution $\tilde{u}^n$. Then we update $\alpha(\tilde{u}^n)$ and 
solve the system again. We keep dionig this untill $\|\tilde{u}^n-\tilde{u}^{n-1}\|\leq\epsilon$ where $\epsilon$ is a predefined tolerance. 
Hopefully this converges to the correct solution, and we can set our $\tilde{u}^n = u^n$. Picard iteration leads us to the following linear system 
for each iteration
\begin{align*}
 \rho\int_\Omega u^nvdx+\Delta t\int_\Omega \alpha(u^{n-1})\nabla u^n\nabla vdx = \rho\int_\Omega u^{n-1}vdx +\Delta t\int_\Omega f(x,t_n)vdx
\end{align*}
we will now insert for $u \simeq \sum\limits_{k}u_k\phi_k$ and because we are working with an undefined number of cells of potentially variable 
size we will transform all the integrals to a reference cell where $-1\leq X\leq 1$. We will also do the calculations in 1 dimension, but they 
easily generalize to multiple dimensions by simply setting each element in say the vector $\mathbf{u}^n$ equal to a vector. The same approach gives 
us a block matrix for M and K.

\begin{align*}
 \rho\sum\limits_{i=1}^{N_x}\int\limits_{-1}^1(\phi_i\phi_j)u^n_idx + \Delta t\sum\limits_{i=1}^{N_x}\int\limits_{-1}^1\alpha(u^{n-1}_i)(\phi'_i\phi'_j)u^ndx = 
  \rho\sum\limits_{i=1}^{N_x}\int\limits_{-1}^1(\phi_i\phi_j)u^{n-1}_idx + \Delta t \sum\limits_{i=1}^{N_x}\int\limits_{-1}^1f_i(x,t_n)\phi_jdx
\end{align*}

If we now set $M_{ij} = \rho\int\limits_0^L(\phi_j\phi_i)dx$ and $K_{ij} = \Delta t\int\limits_0^L\alpha(u^{n-1})(\phi'_j\phi'_i)dx $ we get
\begin{align*}
M_{ij} = \int\limits_0^L(\phi_j\phi_i)dx \implies M_{00} = \int\limits_{-1}^1(\phi_0\phi_0)dx
\end{align*}
We are using P1 elements, and so the only nonzero integrals we will get are when $i=j$ and $i\pm1 = j$, meaning M and K will become tridiagonal.\\
VIS HVA ELEMENTENE BLIR	\\
we are left with
\begin{align*}
M\mathbf{u}^n + K(u^{n-1})\mathbf{u}^n = M\mathbf{u}^{n-1} + \mathbf{b}\\
\implies \mathbf{u}^n = (M+K(u^{n-1}))^{-1}(M\mathbf{u}^{n-1} -\mathbf{b})
\end{align*}
to be solved for each Picard iteration. The crudest Picard iteration will be just one iteration, which is the same as solving the system using
$\alpha(u^{n-1})\simeq\alpha(u^n)$ and this is our chosen approach.
\subsection*{Group Finite Element method}
We can also approximate the nonlinear term by the group finite element method. Looking at the variational form of our equation and inserting 
for $u\simeq \sum\limits_{i=1}^N\phi_iu_i$ we get the following
\begin{align*}
\rho\int_\Omega \sum\limits_i(\phi_i\phi_j)u_i^nd\Omega+\Delta t\int_\Omega \alpha(\sum\limits_i\phi_iu_i^{n})\sum\limits_j(\nabla\phi_j\nabla 
\phi_k)u_j^nd\Omega = \rho\int_\Omega \sum\limits_i(\phi_i\phi_j)u_i^{n-1}d\Omega +\Delta t\int_\Omega f(x,t_n)vd\Omega
\end{align*}
If we now set $\alpha(\sum\limits_i\phi_iu_i^{n})\approx \sum\limits_i\phi_i\alpha(u_i^{n})$ and focus on the integral with $\alpha$, we are left with
(in one dimension)
\begin{align*}
 \int_\Omega \sum\limits_i\phi_i\alpha(u_i^{n})\sum\limits_j(\phi'_j\phi'_k)u_j^ndx
\end{align*}
Which we transfer to a reference element $X\in[-1,1]$ where ${d \phi \over d X} = {d\phi \over dX}{dX \over dx} = {d\phi \over dX}{2\over h}$ 
and ${dx\over dX}={h\over 2}$. Theese calculations will result in an element matrix which we can assemble so that we get a linear system
\begin{align*}
 {(-1)^r(-1)^s\over h}\int\limits_{-1}^1\alpha(u_i)\phi_i\phi'_j\phi'_kdX \\
 \approx{(-1)^r(-1)^s\over 2h}\Big(\underbrace{\sum\limits_r\alpha(u_{i+r})\cdot\phi_{i+r}}_{\neq0 \,\text{when}\, r=1} +
 \underbrace{\sum\limits_s\alpha(u_{i+s})\cdot\phi_{i+s}}_{\neq0 \,\text{when}\, s=0}\Big) \\
 \implies {1\over 2h}(\alpha(u_{i+1})+\alpha(u_i))\cdot\left[\begin{array}{c c}
                                                        1&-1\\
                                                        -1&1
                                                       \end{array}\right] =K^{(i)}
\end{align*}

The entire linear system is then
$$
M\mathbf{u}^n +\Delta tK\mathbf{u}^n = M\mathbf{u}^{n-1} +\mathbf{b}
$$
where M is the same matrix as in (SOMEREFERENCE). This means that the i'th equation in the system is

\subsection*{Newtons Method}
As a different approach we can use Newtons method to estimate the nonlinear term in our equation (\ref{eq}). Newtons method is quite simply 
an iterative method where we linearize our problem through a first order Taylor expansion, and can be expressed as
$$
\mathbf{F}(\mathbf{u}) = 0 \simeq M(u;u^q) = F(u^q) +J(u-u^q)
$$
where $J=\nabla F$ which means that $J_{i,j} = {\d F_i \over \d u_j}$ and $\mathbf{F}(\mathbf{u})$ is the equation we want to solve 
only written on a general form. If we reformulate a bit we find
\begin{align*}
 J(u^{q+1}-u^q) = -F(u^q) \\
 u^{q+1} = u^q-J^{-1}F(u^q)
\end{align*}
which is the very same formulation as in 1D $x^{k+1} = x^k - {f(x^k) \over f'(x^k)}$
We wish to apply this on the variational form of the equation which ends up as our F: (notice that we set $\rho=1$)
\begin{align}
\nonumber
F_i(u) &= \int_\Omega\left(u_i^nv +\Delta t(\alpha(u_i^n)\nabla u_i^n \nabla v) -u_i^{n-1}v -\Delta tf(\mathbf{x},t)v 
 \right)d\Omega \\ \nonumber
 J_{i,j} &= {\d \over \d u_j^n}F_i \\
 J_{i,j} &= \int_\Omega\left(\phi_i\phi_j +\Delta t(\alpha'(u_i^n)\phi_i\nabla u_i^n \nabla \phi_j + 
 \alpha(u_i^n)\nabla \phi_i\nabla\phi_j) \right)d\Omega \label{jacobian}
\end{align}
where we have used $u_i^n = \phi_i u_i^n \implies {\d \over \d u_j^n}u_i^n = \phi_i$, $v=\phi_j$ and ${\d \over \d u_j^n}\alpha(u_i^n) = 
\alpha'(u_i^n)\phi_i$. And there we have the expressions for the entries in the jacobian matrix we can use for a Newtons method 
on our equation. The remainig steps are to assemble the linear problem, and iterate in the same way as we did with Picard iteration. 
Theese are both tasks well suited for the FEniCS software.\\ 
Simplfying (\ref{jacobian}) to one spatial dimension lets us compute the actual entries in the jacobian in a simple manner 
so that we can compare with something else. The simplified version is
\begin{align*}
  J_{i,j} = \int_x\left(\phi_i\phi_j +\Delta t(\alpha'(u_i^n)\phi_i\cdot(u_i^n)'\phi'_j + 
 \alpha(u_i^n)\phi'_i\phi'_j) \right)dx \\
\end{align*}
We notice that we have allready calculated the first and last term and so we can focus on the middle term. Again we will integrate by the trapezoidal rule.
\begin{align*}
 \int\limits_{-1}^1\alpha'(u^n)\phi_i\cdot(u^n)'{2\over h}\phi'_j{h\over2}dX  = {(-1)^r\over h}\Big(\alpha'(u_{i+s}^n)(u_i^n)'\delta_{s_0}\left|_{X=-1} +
 \alpha'(u_{i+s}^n)(u_i^n)'\delta_{s_1}\right|_{X=1}\Big) 
\end{align*}
we approximate the derivative of u by $(u^n_i)' \simeq {1\over2}(u^n_i-u^n_{i+1})$ and tidy up the expression a bit to get
\begin{align*}
 {1\over h}(u^n_i-u^n_{i+1})\left[
 \begin{array}{c c}
  \alpha'(u^n_i) & \alpha'(u^n_{i+1})\\
  -\alpha'(u^n_i) & -\alpha'(u^n_{i+1})
 \end{array}
\right]
\end{align*}
Combining all of the entries then gives us the Jacobian
\begin{align*}
 J_{i,i} &= {1\over h}\left((u^n_i-u^n_{i+1})(\alpha'(u^n_i)-\alpha'(u^n_{i+1})) +(\alpha(u^n_i)+\alpha(u^n_{i+1}))+{2\over3} \right)\\
 J_{i,i+1} &=  {1\over h}\left(-(u^n_i-u^n_{i+1})\alpha'(u^n_{i+1}) +{1\over2}(\alpha(u^n_i)+\alpha(u^n_{i+1}))+{1\over6}\right)\\
 J_{i,i-1} &=  {1\over h}\left((u^n_i-u^n_{i+1})\alpha'(u^n_{i}) +{1\over2}(\alpha(u^n_i)+\alpha(u^n_{i+1}))+{1\over6}\right)
\end{align*}


\section*{Implementation}
To make the program as general as possible, we will make it dimension independent by providing the number of spatial points on the commandline 
as $n_x, n_y, n_z$ and choose a unit interval, square or cube accordingly. Also, even though we will only use one Picard iteration, we will 
implement this as a function where the maximum nuber of iterations is given as an argument. The resulting code can be found in the appendix.

\section*{Verification}
For the first verification of the program we will assume $\rho = \alpha(u) = 1$, $f(\mathbf{x},t) = 0$ and $I(\mathbf{x}) = \cos(\pi x)$. We 
will wrok on the domain $\Omega = [0,1]\times[0,1]$. This should give an analytic solution of $u(x,y,t) = e^{-\pi^2t}\cos(\pi x)$. Since we have used 
the Backward Euler discertization in time we should have an error of $\Delta t$, and therefore also a first order convergence. \\


\subsection*{Errors}
As in any (nummerical) approximation there will be an error in our solution. We can distinguish two types of errors, there is the error we do in 
the approximation of the derivatives, and there is the nummerical error.\\
Our approximation using the Backward Euler scheme has an error which goes like $\mathcal{O}(\Delta t)$ for other reasons the spatial error goes 
like $\mathcal{O}(\Delta x^2)$. Other potential errors which will arise in the FEniCS program are errors due to loss of prescision in the 
nummerical integration we do upon assembly of the matrices M and K. Depending on what kind of nummerical integration we use, the error from integration 
could be $\mathcal{O}(\Delta x)$ for the rectangle (midpoint) rule or it could even be zero if the solution is a polynomial of degree $2n-1$ 
and we integrate by some form of Gaussian quadrature. $n$ is here the order of elements used.
There is allways a possibillity of loosing nummerical precision because of adding a small and a large number since we only have 16 valid decimals.
In our case we also have an obvious error in the approximation of the nonlinear term. 

\section*{Results}

\section*{Final comments}
\end{document}
